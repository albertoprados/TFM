%!TEX TS-program = pdflatex
%!TEX encoding = utf8
\documentclass[12pt, oneside]{book}
\usepackage[T1]{fontenc}
\usepackage[utf8]{inputenc}
\usepackage[english]{babel}
%% FONTS: libertine+biolinum+stix
\usepackage[mono=false]{libertine}
\usepackage[notext]{stix}

% =====================
% = Datos importantes =
% =====================
% hay que rellenar estos datos y luego
% ir a \begin{document}

\title{Uncertainty treatment of discrete elements in Maxwell equations' resolution using the FDTD method.}
\author{Alberto Prados Pérez}
\date{15 Julio 2021}
\newcommand{\tutores}[1]{\newcommand{\guardatutores}{#1}}
\tutores{Dr. Luis Manuel Díaz Angulo}

% ======================
% = Páginas de títulos =
% ======================
\makeatletter
\edef\maintitle{\@title}
\renewcommand\maketitle{%
  \begin{titlepage}
      \vspace*{1.5cm}
      \parskip=0pt
      \Huge\bfseries
      \begin{center}
          \leavevmode\includegraphics[totalheight=6cm]{sello.jpg}\\[2cm]
          \@title
      \end{center}
      \vspace{1cm}
      \begin{center}
          \@author
      \end{center}
  \end{titlepage}
  
  \begin{titlepage}
  \parindent=0pt
  \begin{flushleft}
  \vspace*{1.5mm}
  \setlength\baselineskip{0pt}
  \setlength\parskip{0mm}
  \begin{center}
      \leavevmode\includegraphics[totalheight=4.5cm]{sello.jpg}
  \end{center}
  \end{flushleft}
  \vspace{1cm}
  \bgroup
  \Large \bfseries
  \begin{center}
  \@title
  \end{center}
  \egroup
  \vspace*{.5cm}
  \begin{center}
  \@author
  \end{center}
  \vspace*{1.8cm}
  \begin{flushright}
  \begin{minipage}{8.45cm}
      Memoria del {\bf Trabajo Fin de Máster}.\\ 
      Máster en Física y Matemáticas (FisyMat) \\ 
      University of Granada.

      \vspace*{7.5mm}

      Tutored by:
      % \vspace*{5mm}
  \end{minipage}\par
  \begin{tabularx}{8.45cm}[b]{@{}l}
      \guardatutores
  \end{tabularx}
   \end{flushright}
      \vspace*{\fill}
   \end{titlepage}
   %%% Esto es necesario...
   \pagestyle{tfg}
   \renewcommand{\chaptermark}[1]{\markright{\thechapter.\space ##1}}
   \renewcommand{\sectionmark}[1]{}
   \renewcommand{\subsectionmark}[1]{}
  }
\makeatother

% ======================================
% = Color de la Universidad de Sevilla =
% ======================================
\usepackage{tikz}
\definecolor{USred}{cmyk}{0,1.00,0.65,0.34}

% =========
% = Otros =
% =========
\usepackage[]{tabularx}
\usepackage[]{enumitem}
\setlist{noitemsep}

% ==========================
% = Matemáticas y teoremas =
% ==========================
\usepackage[]{amsmath}
\usepackage[]{amsthm}
\usepackage[]{mathtools}
\usepackage[]{bm}
\usepackage[]{thmtools}
\usepackage{braket}
\usepackage{accents}
\newcommand{\ubar}[1]{\underaccent{\bar}{#1}}

\newcommand{\marcador}{\vrule height 10pt depth 2pt width 2pt \hskip .5em\relax}
\newcommand{\cabeceraespecial}{%
    \color{USred}%
    \normalfont\bfseries}
\declaretheoremstyle[
    spaceabove=\medskipamount,
    spacebelow=\medskipamount,
    headfont=\cabeceraespecial\marcador,
    notefont=\cabeceraespecial,
    notebraces={(}{)},
    bodyfont=\normalfont\itshape,
    postheadspace=1em,
    numberwithin=chapter,
    headindent=0pt,
    headpunct={.}
    ]{importante}
\declaretheoremstyle[
    spaceabove=\medskipamount,
    spacebelow=\medskipamount,
    headfont=\normalfont\itshape\color{USred},
    notefont=\normalfont,
    notebraces={(}{)},
    bodyfont=\normalfont,
    postheadspace=1em,
    numberwithin=chapter,
    headindent=0pt,
    headpunct={.}
    ]{normal}
\declaretheoremstyle[
    spaceabove=\medskipamount,
    spacebelow=\medskipamount,
    headfont=\normalfont\itshape\color{USred},
    notefont=\normalfont,
    notebraces={(}{)},
    bodyfont=\normalfont,
    postheadspace=1em,
    headindent=0pt,
    headpunct={.},
    numbered=no,
    qed=\color{USred}\marcador
    ]{demostracion}

% Los nombres de los enunciados. Añade los que necesites.
\declaretheorem[name=Observaci\'on, style=normal]{remark}
\declaretheorem[name=Corolario, style=normal]{corollary}
\declaretheorem[name=Proposici\'on, style=normal]{proposition}
\declaretheorem[name=Lema, style=normal]{lemma}

\declaretheorem[name=Teorema, style=importante]{theorem}
\declaretheorem[name=Definici\'on, style=importante]{definition}

\let\proof=\undefined
\declaretheorem[name=Demostraci\'on, style=demostracion]{proof}


% ============================
% = Composición de la página =
% ============================
\usepackage[
    a4paper,
    textwidth=80ex,
]{geometry}

\linespread{1.069}
\parskip=10pt plus 1pt minus .5pt
\frenchspacing
% \raggedright


% ==============================
% = Composición de los títulos =
% ==============================

\usepackage[explicit]{titlesec}

\newcommand{\hsp}{\hspace{20pt}}
\titleformat{\chapter}[hang]
    {\Huge\sffamily\bfseries}
    {\thechapter\hsp\textcolor{USred}{\vrule width 2pt}\hsp}{0pt}
    {#1}
\titleformat{\section}
  {\normalfont\Large\sffamily\bfseries}{\thesection\space\space}
  {1ex}
  {#1}
\titleformat{\subsection}
  {\normalfont\large\sffamily}{\thesubsection\space\space}
  {1ex}
  {#1}

% =======================
% = Cabeceras de página =
% =======================
\usepackage[]{fancyhdr}
\usepackage[]{emptypage}
\fancypagestyle{plain}{%
    \fancyhf{}%
    \renewcommand{\headrulewidth}{0pt}
    \renewcommand{\footrulewidth}{0pt}
}
\fancypagestyle{tfg}{%
    \fancyhf{}%
    \renewcommand{\headrulewidth}{0pt}
    \renewcommand{\footrulewidth}{0pt}
    \fancyhead[LE]{{\normalsize\color{USred}\bfseries\thepage}\quad
                    \small\textsc{\MakeLowercase{\maintitle}}}
    \fancyhead[RO]{\small\textsc{\MakeLowercase{\rightmark}}%
                    \quad{\normalsize\bfseries\color{USred}\thepage}}%
                    }

% =============================
% = El documento empieza aquí =
% =============================
\begin{document}


\maketitle

\frontmatter
\tableofcontents

\mainmatter

%========================
\chapter*{English Abstract}
\addcontentsline{toc}{chapter}{English Abstract}
\markright{English Abstract}



\begin{otherlanguage}{english}
    Analysis of subcell phenomenology in the Yee algorithm using the FDTD method...
\end{otherlanguage}

\chapter{Introduction}
In this work an analysis of the Stochastic FDTD method is presented. The S-FDTD method is an approach for computing the variance in the electromagnetic fields due to the uncertainties in electrical properties. Those may vary because of uncertainty in the measurements, variations in materials, variations in manufacturing, etc.
The traditional FDTD method uses electrical properties such as permittivity and conductivity as inputs and return the average values of the fields. So, if our goal is to compute other statistical quantities such as the variance of the field, although other statistical properties may also be of interest such as $90\%$ confidence intervals, before the development of the S-FDTD we would be bound to use the Monte Carlo method. This implies running a huge number of simulations to obtain statistics for the fields, this can be made in one dimensional simulations but would be impractical for three dimensional ones.

The S-FDTD takes the same two equations of the FDTD method as a starting point, namely the discretized version of Faraday's and Ampere's law. Next, the delta method, which can be seen as a Taylor's expansion, is used to derive expressions for the mean and variance values. The mean values for this method are the same as those obtained with the FDTD as expected; to get this, higher order terms are neglected in the expansion. For the variance, a series of approximations are made as well, and the procedure is much more complex. At the end of the method two expressions are derived for the electric and magnetic field. In this expressions appear some unknown quantities, those are the correlation coefficients between the electrical properties with the fields. 

The main goal of this work is to explore this correlations. In principle we can not know anything else that they are between $-1$ and $1$. These coefficients vary with time and a model to try to find them before or within making any simulations have not been proposed and this question could have no answer. Until now, several approximations have been tried to estimate the correlation coefficients, for example, assuming a value of one, which overestimate the result given by the Monte Carlo method. Therefore, this work will seek to understand the nature of this correlations which could be the first step to propose an accurate method to compute these coefficients.








\chapter{Introduction to the FDTD method.}

The numerical FDTD method (``finite difference time domain") enables the equations in partial derivatives resolution. As a first step, this method will be applied to one of the simplest equations, the one dimensional wave equation:
\begin{equation}
\frac{\partial^2 u}{\partial t^2}= c^2 \frac{\partial^2 u}{\partial x^2}.
\end{equation}
The analytic solution of this equation is known and it is just the combination of two functions $F$ and $G$ in the following way:
\begin{equation}
u(x,t)= F(x+ct)+G(x-ct).
\end{equation}
\indent Now that we know this, finite differences will be computed. First, the following function will be expanded as a Taylor series, $u(x,t_n)$, about the point $x_i$ to point $x_i+\Delta x$, keeping time fixed [Allen Taflove,Susan C.Hagness]:
\begin{equation}
u(x_i+\Delta x)|_{t_n} = u|_{x_i,t_n} + \Delta x \frac{\partial u}{\partial x} \bigg |_{x_i,\: t_n} + \frac{\left(\Delta x\right)^2}{2}\frac{\partial^2 u}{\partial x^2}\bigg |_{x_i,\: t_n} +\frac{\left(\Delta x\right)^3}{3!}\frac{\partial^3 u}{\partial x^3}\bigg |_{x_i,\: t_n} + \frac{\left(\Delta x\right)^4}{4!}\frac{\partial^4 u}{\partial x^4}\bigg |_{\xi_1,\: t_n},
\end{equation} 
where the last term is the error term and point $\xi_1$ is located in the interval $(x_i, x_i+\Delta x)$. If series expansion is carried about point $x_i - \Delta x$, and both expresions are added, it remains:
\begin{equation}
u(x_i+\Delta x) \bigg |_{t_n}+u(x_i-\Delta x) \bigg |_{t_n}=2u|_{x_i,t_n}+\left(\Delta x\right)^2\frac{\partial^2 u}{\partial x^2}\bigg |_{x_i,\: t_n} + O\left[ \left( \Delta x\right)^2\right],
\end{equation}
the last term being the error...


\chapter{Maxwell's equations and the Yee algorithm.}

Maxwell's equations in their general form are written in three dimensions as...



\section{One dimensional Maxwell's equations.}
One dimensional Maxwell's equations, in free space, are reduced to:

\begin{align}
& \frac{\partial E_x}{\partial t}=-\frac{1}{\epsilon_0}  \frac{\partial H_y}{\partial z}, \\
& \frac{\partial H_y}{\partial t}=-\frac{1}{\mu_0} \frac{\partial E_x}{\partial z}.
\end{align}

These equations are those of a plane wave propagating in the $z$ direction with the electric field in the $x$ direction and the magnetic field oriented in the $y$ direction. Using the central difference approximation in the spatial and time derivatives, the following form of the equations is obtained:

\begin{align}
 \frac{E_x^{n+\frac{1}{2}}(k)-E_x^{n-\frac{1}{2}}(k)}{\Delta t}=-\frac{1}{\epsilon_0}\frac{H_y^n \left(k+\frac{1}{2}\right) - H_y^n\left(k-\frac{1}{2}\right)}{\Delta x} \\
\frac{H_y^{n+1} \left(k+\frac{1}{2}\right) -H_y^{n}\left( k +\frac{1}{2}\right)}{\Delta t}=-\frac{1}{\mu_0}\frac{E_x^{n+\frac{1}{2}} \left(k+1\right) - E_x^{n+\frac{1}{2}}\left(k\right)}{\Delta x}.
\end{align}

...
\subsection{Propagation in a dielectric medium.}


\subsection{Propagation in a lossy dielectric medium.}


\section{Numerical Dispersion.}
[Book EM simulation techniques based on the FDTD method]
\section{Stability in the FDTD method.}
Courant condition.[Book EM simulation techniques based on the FDTD method]


\chapter{Source types (Hard source/Soft Source).}


\chapter{Stochastic FDTD.}
\section{Introduction to the S-FDTD method.}
In the FDTD simulations considered in this work until this point, the values of the permittivity and conductivity used were the average values of the materials and therefore the values of the fields obtained were also their average values. However, materials are statistically variable and this have to be considered since the fields obtained will also vary statistically. \\
As explained in the last paragraph [Stochastic FDTD for analysis of Statistical Variation in EM], a method is needed to compute these statistical quantities such as variance or standard variation. \\
The classical method used to compute the statistical properties is the Monte Carlo method [PereColet], but this method comes under one big issue, it requires a huge number of simulations with properties selected at random.
Now, the Stochastic FDTD (S-FDTD) method is presented which provides a estimate of the mean and variation of the fields at every point in space and time. \\
The usual FDTD equations adapted in pseudo code format (equation number) are complemented with additional equations to compute the standard deviation. Previously, the electrical properties $\epsilon_r$, $\ubar{\sigma}$ (where the symbol for conductivity has been changed to avoid confusion with the standard deviation) were the average values. \\
So the equations (1y2 en[S-FDTD]) must be thought as stochastic equations with four random variables, namely $E_x$, $H_y$, $\epsilon_r$ and $\ubar{\sigma}$.
\section{Mean Field Values}
The delta method [Delta method] will be the tool used to compute the mean and variance of the fields. Take a function of stochastic variables and expand it in Taylor's series, next apply the expectation operator and neglect higher order terms getting the following expression for the expectation [S-FDTD]:
\begin{equation}
E\lbrace g(x_1,x_2,...,x_n)\rbrace \approx E\lbrace g(\mu_{x_1},\mu_{x_2},...,\mu_{x_n})\rbrace + E \left\lbrace \sum_{i=1}^n \frac{\partial g}{\partial x_i}\bigg |_{\mathrlap{\mu_{x_1},\mu_{x_2}...\mu_{x_n}}}(x_i-\mu_{x_i}) \right\rbrace + ...
\end{equation}
where higher order terms were neglected. Now, using that $E\lbrace x_i\rbrace =\mu_{x_i}$ gives:
\begin{equation}
E\lbrace g(x_1,x_2,...,x_n)\rbrace \approx E\lbrace g(\mu_{x_1},\mu_{x_2},...,\mu_{x_n})\rbrace
\end{equation}
This equation tells that the expectation of the fields is obtained using the average of the variables in the fields equations(1y2). Therefore in the S-FDTD method the fields $E_x$ and $H_y$ will be the same as in the traditional FDTD scheme.
\section{Variance of H fields}
For the variance of the fields we start again with the definition:
\begin{equation}
\sigma^2 \lbrace g(x_1,x_2,...,x_n)\rbrace = E\lbrace g(x_1,x_2,...,x_n)^2\rbrace - E\lbrace g(x_1,x_2,...,x_n)\rbrace ^2,
\end{equation}
and now the Taylor expansion is applied obtaining:
\begin{equation}
\sigma^2 \lbrace g(x_1,x_2,...,x_n)\rbrace \approx \sum_{i=1}^n\sum_{j=1}^n \frac{\partial g}{\partial x_i}\frac{\partial g}{\partial x_j} \bigg |_{\mathrlap{\mu_{x_1},\mu_{x_2}...\mu_{x_n}}} E\lbrace(x_i-\mu_{x_i})(x_j-\mu_{x_j})\rbrace,
\end{equation}
where the first term in the expansion of $E\lbrace g(x_1,x_2,...,x_n)^2\rbrace$ cancels out with the expansion of $E\lbrace g(x_1,x_2,...,x_n)\rbrace ^2$ and, again, higher order terms have been neglected.
So the goal is to compute the variance for the equation (1), for this purpose two relations involving the variance will be needed, namely:
\begin{equation}\label{Relations for the variance}
\begin{split}
\sigma^2\left[X\pm Y\right] & =\sigma_X^2+\sigma_Y^2\pm 2\textit{Cov}(X,Y), \\
\sigma^2\lbrace aX \rbrace  & = a^2 \sigma^2 \lbrace X \rbrace,
\end{split}
\end{equation}
as well as the identity for the covariance $\textit{Cov}(X,Y)=\rho_{XY}\sigma\lbrace X \rbrace \sigma\lbrace Y\rbrace$. \\
\indent The variance of the \textit{H} field equation is:
\begin{equation}
\begin{split}
\sigma^2\left\lbrace H_y^{n+1/2}(k+1/2)-H_y^{n-1/2}(k+1/2)\right\rbrace = \frac{\Delta t^2}{\mu^2 \Delta z^2}\sigma^2\left\lbrace E_x^n(k+1)-E_x^n(k)\right\rbrace
\end{split}.
\end{equation}
This equation can be worked out using the relations \ref{Relations for the variance}, getting the result:
\begin{equation}
\begin{split}
& \sigma^2\left\lbrace H_y^{n+1/2}(k+1/2)\right\rbrace-2\sigma\left\lbrace H_y^{n+1/2}(k+1/2)\right\rbrace \sigma\left\lbrace H_y^{n-1/2}(k+1/2)\right\rbrace \\
&+ \sigma^2\left\lbrace H_y^{n+1/2}(k+1/2)\right\rbrace=\frac{\Delta t^2}{\mu^2 \Delta z^2} \left(\sigma^2\left\lbrace E_x^n(k+1)\right\rbrace \right. \\
& \left. -2\sigma\left\lbrace E_x^n(k+1)\right\rbrace \sigma\left\lbrace E_x^n(k)\right\rbrace + \sigma^2 \left\lbrace E_x^n(k)\right\rbrace \right).
\end{split}
\end{equation}
In this equation an approximation has been made, the correlation coefficients have been set equal to one. The correlation coefficient varies between $-1$ and $1$. This coefficient is $\rho_{H_{k+1/2}^{n+1/2},H_{k+1/2}^{n-1/2}}$ for the \textit{H} field, so it is relating the values of the \textit{H} field separated by one time step $\Delta t$ which justifies the approximation. For the correlation coefficient of the \textit{E} field an analogous treatment is made, it's just the correlation between the values of \textit{E} field separated by one spatial step $\Delta z$ [24 en el artículo]. \\
\indent Observing the last expression for the variance it can be noted that there is a perfect square in each side, taking the square root yields the following update equation for the variance of the \textit{H} field:
\begin{equation}
\sigma\left\lbrace H_y^{n+1/2}(k+1/2) \right\rbrace = \sigma\left\lbrace H_y^{n-1/2}(k+1/2)\right\rbrace + \frac{\Delta t}{\mu \Delta z}\left(\sigma \left\lbrace E_x^n(k+1)\right\rbrace -\sigma\left\lbrace E_x^n(k)\right\rbrace \right)
\end{equation}
\section{Variance of E fields}

\section{MPI implementation}
MPI is a library used for parallel processing [Book EM simulation techniques based on the FDTD method]


\section{Standard Deviation in Reflectance and Transmittance}
With the previous formalism, it has been possible to obtain the standard deviation of the fields. Now, the standard deviation of the reflection and transmission coefficient is sought. To perform this task, the propagation of errors is going to be employed. Given a quantity $X$, which is a function of other variables $X=f(x_1,x_2,...)$, the variance of $X$, in case $f$ is a linear combination of the variables $\left( f= \sum_i^n a_i x_i \right)$, is [John R.Taylor An introduction to error analysis]:
\begin{equation}\label{error propagation variance}
\sigma_X ^2 = \sum_i ^n a_i^2 \sigma_i^2 + \sum_i^n \sum_{j(j\neq i)} ^n a_i a_j \rho_{ij} \sigma_i \sigma_j .
\end{equation}
Applying this to the reflection coefficient 
\begin{equation}
R=\frac{E_{ref}(\omega)}{E_{src}(\omega)} ,
\end{equation}
where $E_{src}$ is just the source field, the $\sigma_R$ will be computed (after that, the modulus of $R$ and $\sigma_R$ will be obtained for the representation). The source field has no standard deviation so it will be treated as a constant. The first step is to write $\sigma_R$ as
\begin{equation} \label{error RyT}
\sigma_R= \frac{\partial R}{\partial E_{ref}(\omega)} \sigma_{E_{ref}(\omega)}
\end{equation} 
Now, as $E_{ref}(\omega)$ is obtained using the discrete Fourier transform (DFT), which is a linear combination of the electric fields in time space, as the next expression shows, equation \ref{error propagation variance} can be applied.
\begin{equation}
E_k(\omega)=\sum_{n=0}^{N-1} e^{-i\frac{2\pi}{N} k n} E_n(t).
\end{equation} 
The coefficients of the Fourier transform can be identified with the $a's$ in \ref{error propagation variance}, where the second term is the covariance written in terms of the correlation $\rho_{ij}$. In the S-FDTD formalism the correlations between electric fields at different times is high and it is assumed to be one [24(Stochastic FDTD Phd)]. Using this, expression \ref{error propagation variance} can be written as a perfect square using the multinomial theorem:
\begin{equation}
\sigma_X=\sum_i ^n a_i \sigma_i,
\end{equation}
so the standard deviation of the Fourier transform of the fields is the Fourier transform of the standard deviations, which could have also been viewed as the linearity in the Fourier Transform.
Finally, the standard deviation of the coefficients $R$ and, in an analogous way, $T$, is obtained with \ref{error RyT}.

\chapter{Monte Carlo FDTD.}
\section{Introduction to the M-FDTD method.}
\indent Monte Carlo method is the traditional approach to evaluate statistical quantities such as the mean and variance. This method is well known and widely used in several areas of physics. In the case at hand, the Monte Carlo is applied to the usual FDTD algorithm explained before. \\
\indent The M-FDTD algorithm will be used to compare the results obtained with the S-FDTD. The differences will depend mainly in the choice of the correlation coefficients in the S-FDTD variance equations, therefore the Monte Carlo can be used to check the validity of a particular choice or even to estimate the value of the correlations [S-FDTD accuracy improvement]. \\
\indent The first step for the M-FDTD is to generate the parameters, namely the permittivity and conductivity, randomly distributed according to a given distribution. The Gaussian distribution have been chosen for the main results although other distributions could be used to produce the electrical parameters, for this purpose the Metropolis algorithm can be employed. Variability could also be introduced in the thickness of the materials but this will not be approached in this work. \\
\indent Variability can be present in materials for several reasons. Consider some inhomogeneous materials such as biological tissues like skin, fat or muscle. The composition and therefore the electrical properties will vary among people, for example, tissues' conductivities will depend on the level of hydration of the individual.
Another way in which variability can appear is due to measurement uncertainties. In the next section, two different schemes for computing the Monte Carlo will be explored to treat the two sources of uncertainties explained here. 

\section{Scheme of the Monte Carlo method.}








%\chapter{Los enunciados}

%\section{Teoremas y demostraciones}


%\begin{theorem}[Euclides]\label{thm:th1}
   % Esto es un Teorema. Se numeran a partir del 1 en cada capítulo. Como son importantes, %tienen un cuadrado rojo al principio. Llevan letra cursiva.
%\end{theorem}

%\begin{proof}
%    Esto es la demostración. Al final de la demostración se puede ver un cuadrado rojo %similar al de los teoremas. Las demostraciones no llevan letra cursiva.
%\end{proof}


%\begin{definition}\label{def:1}
%    Esto es una definición. Las definiciones son importantes; también llevan un %cuadradito rojo.
%\end{definition}


%\subsection{Otros enunciados}


%\begin{remark}
%    Esto es una observación, que dice que $e=mc^{2}$. Como las observaciones no son %importantes, no llevan cuadrado rojo, y el tipo de letra no es cursiva.
%\end{remark}


%\begin{proof}
    %Si la demostración acaba en una fórmula, para poner el cuadrado rojo a la altura de %la última formula, hay que usar la orden \verb|\qedhere|, como en este caso:
%    \[
%        e=mc^{2}.\qedhere
%   \]

%\end{proof}


%\begin{corollary}\label{cor:1}
  %  Esto es un corolario.
%\end{corollary}

%\begin{proposition}\label{pro:1}
%   Esto es una proposición.
%\end{proposition}

%\begin{lemma}[Gauss]\label{lem:1}
    %Esto es un lema.
%\end{lemma}


%\backmatter

%\bibliographystyle{acm}
% \biliography{miarchivo} % -> lee miarchivo.bib
%===============


\end{document}